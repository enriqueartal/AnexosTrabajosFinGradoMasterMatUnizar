


%%%%%%%%%%%%%%% Preasignado
\newcommand\preasignado{\eleccionNO}

%%%%%%%%%%%%%%% Lugar
\newcommand\lugarciencias{\eleccionSI}
\newcommand\lugarotro{}


%%%%%%%%%%%%%% Texto descripción
\newcommand\descripcion{
Uno de los objetivos de este trabajo es que el estudiante entienda la estructura del cuerpo de funciones
meromorfas de una superficie de Riemann compacta y las consecuencias que tiene dicha estructura, en
particular los Teoremas de finitud de la Cohomología de haces sobre superficies de Riemann. Para ello hay que
estudiar nociones de álgebra homológica, variable compleja, topología, teoría de haces (sheaves) hasta llegar a
probar que toda superficie de Riemann compacta es algebraica y estudiar las propiedades de sus divisores. El
estudiante combinará trabajo personal con discusiones con el director.
\relleno{0}{0}} % Los parámetros son para las dimensiones de la caja.

%%%%%%%%%%%%%%%%%%% Descripción de actividades y horas
\newcommand{\acti}{}
\newcommand{\horasi}{}
\newcommand{\actii}{}
\newcommand{\horasii}{}
\newcommand{\actiii}{}
\newcommand{\horasiii}{}
\newcommand{\activ}{}
\newcommand{\horasiv}{}
\newcommand{\actv}{}
\newcommand{\horasv}{}
\newcommand{\actvi}{}
\newcommand{\horasvi}{}
\newcommand{\actvii}{}
\newcommand{\horasvii}{}
\newcommand{\actviii}{}
\newcommand{\horasviii}{}
%%%%%%%%%%%%%%%%%%%%%%%%%%%%%%% Total horas
\newcommand{\horastotal}{}
%%%%%%%%%%%%%%%%%%

%%%%%%%%%%%Fechas Anexo I
\newcommand{\diaI}{%
%\makebox[1cm][l]{
20
%}
}
\newcommand{\mesI}{%
%\makebox[4cm][l]{
julio
%}
}
\newcommand{\anyoI}{%
%\makebox[2cm][l]{
25
%}
}